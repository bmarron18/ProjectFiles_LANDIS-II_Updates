%%%%%%%%%%%%%%%%%%%%%%%%%%%%%%%%%%%%%%%%%

% Project Title: LANDIS-II SOFTWARE ENHANCEMENT
% Grant/Agreement Number: 14-JV-11242313-112
% Cooperator Name: PORTLAND STATE UNIVERSITY
% Performance Start Date: 08/05/2014
% Performance End Date: 12/31/2016
% Report Source: Cooperator
% Report Due Date: 03/31/2017


% Internal Project Title: LANDIS_Upgrades_Project
% Internal Project Number: 2016SoE021
% Original: bmarron, 07 Apr 2017
%
%%%%%%%%%%%%%%%%%%%%%%%%%%%%%%%%%%%%%%%%%

\documentclass[letterpaper,11pt]{texMemo}

\usepackage[utf8]{inputenc}
\usepackage{graphicx}
\usepackage{hyperref}
\usepackage{amsmath}
\usepackage{amsthm}
\usepackage{amssymb}
\usepackage{apacite}
\usepackage[english]{babel}
\usepackage{comment}
\usepackage{csquotes}

\setlength{\parindent}{0pt}

\memoto{Eric Gustafson}
\memofrom{Rob Scheller}
\memosubject{Progress report on LANDIS-II Software Enhancement (Grant 14-JV-11242313-112)}
\memodate{\today}
\logo{\includegraphics[width=0.35\textwidth]{graphics/psulogo_horiz_msword.jpeg}}

\begin{document}
\maketitle
This memo provides a concise, update summary of the status of the tasks assigned to the LANDIS-II Software Enhancement Project (\enquote{Project}). Specific and general tasks as well as their priorities are defined by the document dated September 7, 2016.  The updates for Top Priority Tasks and Secondary Priority tasks reference the task numbers in the September 7, 2016 document. A milestones update and a cost accounting update also are provided.\\

\section*{Top Priority Tasks}
\begin{description}
  \item[Task 1 -- Integrate climate library into appropriate extensions] Both NECN succession and Biomass succession now use climate data. Biomass harvest executes seamlessly with NECN succession and Biomass succession. This task is 100\% complete.
  \item[Task 2 -- Re-engineer the Site Tool] ???
  \item[Task 3 -- Add Metadata library for LANDVIZ compatibility to all extensions] This task is roughly 80\% complete.
  \item[Task 4 -- Develop R scripts to run the model and generate LANDVIZ web sites] This task is slated to begin after Task 3 has been completed.
  \item[Task 5 -- Provide enhancements and bug fixes for LANDVIZ] LANDVIZ now accepts non-integer hectare area for input rasters. The PreProcTool and its Python dependencies (including PyInstaller) have been updated to allow for rebuilds of LANDVIZ. A step-by-step rebuild protocol has been defined and documented. This task is 100\% complete.
  \item[Task 6 -- Fix remaining bugs in Harvest] This task is underway and is about 50\% complete.
  \item[Task 7 -- Implement a prescribed burning option] This task needs some discussion about the best approach before being assigned to a programmer; task not attempted.
  \item[Task 8 -- Perform maintenance and add enhancements to Core Model] Rebuilding the Core Model is now a streamlined and well-documented five-stage process. Core Model enhancements include adding the names of all input files to the log, writing all extension versions to log, providing species name in error messages, providing the version that is installed, and replacing “Ageing cohorts” with “Growing cohorts” in console. This task is 100\% complete.
\end{description}

\section*{Secondary Priority Tasks}
\begin{description}
  \item[Task 9 -- Integrate Forest Service vegetation data] This task needs more information; task not attempted.
  \item[Task 10 -- Make Landscape Builder compatible with biomass succession extensions] ???
  \item[Task 11 -- Improve error messages (more informative) throughout LANDIS-II] This task needs very specific suggestions; task not attempted.
  \item[Task 12 -- Develop tools to build all input maps] This task is a rather huge project and would require another \$100k; task not attempted.
  \item[Task 13 -- Revise all UGs for clarity and currency.] This task is the job of each individual developer; task not attempted. We have performed major restructuring and standardization on many repositories which could be used to provide guidelines for a ‘standard’ UG format (see below), but our rules dictate that we cannot revise without permission.
  \item[Task 14 -- Develop links between FVS and LANDIS-II ] This task is a worthy idea but is not programmer-ready; task not attempted.
  \item[Task 15 -- Develop outputs for NFS] This task is a worthy idea but is not programmer-ready; task not attempted.
  \item[Task 16 -- Enable use of American fuel models in the Dynamic Fire and Fuels extensions] This task is a worthy idea but is not programmer-ready; task not attempted.
\end{description}

\section*{Supplemental Tasks}
In addition to the above, the following supplemental tasks were included in the Project.

\begin{enumerate}
  \item Major restructuring and standardization of all repositories under Rob Scheller. The restructuring and standardization includes:
\begin{itemize}
  \item Standardization of the Visual Studio .dll build process through a) modification of .csproj files, b) installation of .cmd files, and c) referencing the Support-Library-DLLs repository. The standardization provides simple, error-free extension builds in Visual Studio.
  \item Standardization of .iss files for error-free building of extension installers in Inno Setup.
  \item Standardization and clean-up of all extension-installed scenarios for error-free LANDIS-II runs of examples.
  \item Standardization of repository structure for consistency and ease of access to repository materials.
  \item Complete revision and standardization of the README files.
\end{itemize}
The major restructuring and standardization task is roughly 75\% complete.

  \item A variety of small updates and fixes throughout the LANDIS-II Foundation repositories.
\end{enumerate}

\section*{Milestones}
Updates and enhancements to output extensions are projected to be completed by May 1, 2017. \\

Updates and enhancements to succession extensions are projected to be completed by May 15, 2017.\\

Updates and enhancements to disturbance extensions are projected to be completed by May 30, 2017.


\section*{Cost Accounting}
Programmer and project management costs to-date total \$17,100.00.


\end{document}
